\documentclass{article}

\usepackage[english]{babel}
\usepackage[utf8]{inputenc}
\usepackage{amsmath,amssymb}
\usepackage{parskip}
\usepackage{graphicx}

% Margins
\usepackage[top=2.5cm, left=3cm, right=3cm, bottom=4.0cm]{geometry}
% Colour table cells
\usepackage[table]{xcolor}

% Get larger line spacing in table
\newcommand{\tablespace}{\\[1.25mm]}
\newcommand\Tstrut{\rule{0pt}{2.6ex}}         % = `top' strut
\newcommand\tstrut{\rule{0pt}{2.0ex}}         % = `top' strut
\newcommand\Bstrut{\rule[-0.9ex]{0pt}{0pt}}   % = `bottom' strut

%%%%%%%%%%%%%%%%%
%     Title     %
%%%%%%%%%%%%%%%%%
\title{Matrix Theory EE5609 \\ Assignment-1}
\author{Prasanth Kumar Duba \\ EE20RESCH11008}
\date{\today}

\begin{document}
\maketitle

%%%%%%%%%%%%%%%%%
%   Problem 1   %
%%%%%%%%%%%%%%%%%
\section{Problem 53: Find the direction in which a straight line must be drawn through the point (\begin{array}{c}
-1\\
2\\
\end{array}) so that its point of intersection with the line $$ (1 \hspace{10} 1)x = 4 $$ may be the distance of 3 units from this point.}
Solution:\newline \tablespace
Let the slope of the line m, which is passing through the point A(-1,2).\newline \tablespace So, the equation of the line is : 
\begin{align}
    \label{eq:example_equation} % Equation label; can be used for referencing
    y = mx+m+2 \implies (m \hspace{10}-1)x = -m-2
\end{align}
Also the given equation of the line is:
\begin{align}
    \label{eq:example_equation}
    y=4-x \implies(1 \hspace{10}1) x = 4
\end{align}
Consider these two lines meet at a point B.
\newline
\tablespace From (1) and (2), $$B=(\frac{m+6}{m+1}, \frac{3m-2}{m+1})$$ 
\newline Now, Given that AB = 3
\newline \tablespace \implies AB^2 = 9 \implies  $$(\frac{m+6}{m+1}+1)^2 $$ + $$(\frac{3m-2}{m+1}-2)^2=9 $$
\newline \tablespace \implies $$2m^2-m-28=0$$
\newline \tablespace \implies $$m=4, m=-3.5$$
\newline \tablespace \implies $$tan\theta =4 \hspace{3} or \hspace{3} tan\theta= -3.5   $$ \newline \tablespace Hence,\hspace{3} the  \hspace{3} direction \hspace{3} angle \hspace{3} $$\theta = 75.96^{\circ} $$ \hspace{3} or \hspace{3} $$\theta = -74.05^{\circ} $$   
\newline \tablespace Now, \hspace{3}consider \hspace{3} m=4 \hspace{3} and \hspace{3}the\hspace{3} equations\hspace{3} are: 

$$    4x-y+6=0 $$ $$    x+y-4=0 $$ 
\newline \tablespace The plot is as follows:
% Example of how to add figure (can be used for jpeg, png, pdf, eps etc)
\begin{figure}[h]
\includegraphics[width=15cm]{1.jpg}
\end{figure}

\end{document}
